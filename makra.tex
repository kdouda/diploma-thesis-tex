%%% Tento soubor obsahuje definice různých užitečných maker a prostředí %%%
%%% Další makra připisujte sem, ať nepřekáží v ostatních souborech.     %%%

\usepackage[a-2u]{pdfx}     % výsledné PDF bude ve standardu PDF/A-2u

\usepackage{ifpdf}
\usepackage{ifxetex}
\usepackage{ifluatex}

%%% Nastavení pro použití samostatné bibliografické databáze.
\usepackage[
   backend=bibtex
  ,style=iso-authoryear
%  ,style=iso-numeric
  ,sortlocale=cs_CZ
  ,alldates=iso
  ,bibencoding=UTF8
  %,block=ragged
]{biblatex}
\let\cite\parencite
\bibliography{literatura}

%% Přepneme na českou sazbu, fonty Latin Modern a kódování češtiny
\ifthenelse{\boolean{xetex}\OR\boolean{luatex}}
   { % use fontspec and OpenType fonts with utf8 engines
			\usepackage[english,slovak,czech]{babel}
			\usepackage[autostyle,english=british,czech=quotes]{csquotes}
			\usepackage{fontspec}
			\defaultfontfeatures{Ligatures=TeX,Scale=MatchLowercase}
   }
   {
			\usepackage[english,slovak,czech]{babel}
			\usepackage{lmodern}
			\usepackage[T1]{fontenc}
			\usepackage{textcomp}
			\usepackage[utf8]{inputenc}
			\usepackage[autostyle,english=british,czech=quotes]{csquotes}
	 }
\ifluatex
\makeatletter
\let\pdfstrcmp\pdf@strcmp
\makeatother
\fi

%%% Další užitečné balíčky (jsou součástí běžných distribucí LaTeXu)
\usepackage{amsmath}        % rozšíření pro sazbu matematiky
\usepackage{amsfonts}       % matematické fonty
\usepackage{amssymb}        % symboly
\usepackage{amsthm}         % sazba vět, definic apod.
\usepackage{bm}             % tučné symboly (příkaz \bm)
\usepackage{graphicx}       % vkládání obrázků
\usepackage{listings}       % vylepšené prostředí pro strojové písmo
\usepackage{fancyhdr}       % prostředí pohodlnější nastavení hlavy a paty stránek
\usepackage{icomma}         % inteligetní čárka v matematickém módu
\usepackage{dcolumn}        % lepší zarovnání sloupců v tabulkách
\usepackage{booktabs}       % lepší vodorovné linky v tabulkách
\makeatletter
\@ifpackageloaded{xcolor}{
   \@ifpackagewith{xcolor}{usenames}{}{\PassOptionsToPackage{usenames}{xcolor}}
  }{\usepackage[usenames]{xcolor}} % barevná sazba
\makeatother
\usepackage{multicol}       % práce s více sloupci na stránce
\usepackage{caption}
\usepackage{enumitem}
\setlist[itemize]{noitemsep, topsep=0pt, partopsep=0pt}
\setlist[enumerate]{noitemsep, topsep=0pt, partopsep=0pt}
\setlist[description]{noitemsep, topsep=0pt, partopsep=0pt}

\usepackage{tocloft}
\setlength\cftparskip{0pt}
\setlength\cftbeforechapskip{1.5ex}
\setlength\cftfigindent{0pt}
\setlength\cfttabindent{0pt}
\setlength\cftbeforeloftitleskip{0pt}
\setlength\cftbeforelottitleskip{0pt}
\setlength\cftbeforetoctitleskip{0pt}
\renewcommand{\cftlottitlefont}{\Huge\bfseries\sffamily}
\renewcommand{\cftloftitlefont}{\Huge\bfseries\sffamily}
\renewcommand{\cfttoctitlefont}{\Huge\bfseries\sffamily}

% vyznaceni odstavcu
\parindent=0pt
\parskip=11pt

% zakaz vdov a sirotku - jednoradkovych pocatku ci koncu odstavcu na prechodu mezi strankami
\clubpenalty=1000
\widowpenalty=1000
\displaywidowpenalty=1000

% nastaveni radkovani
\renewcommand{\baselinestretch}{1.20}

% nastaveni pro nadpisy - tucne a bezpatkove
\usepackage{sectsty}    
\allsectionsfont{\sffamily}

% nastavení hlavy a paty stránek
\fancyhf{}
\fancyhead[RO,LE]{\rightmark}
\fancyfoot[RO,LE]{\thepage}
\renewcommand{\footrulewidth}{.5pt}
\fancypagestyle{plain}{%
\fancyhf{} % clear all header and footer fields
\fancyfoot[RO,LE]{\thepage}
\renewcommand{\headrulewidth}{0pt}
\renewcommand{\footrulewidth}{0.5pt}}

% Tato makra přesvědčují mírně ošklivým trikem LaTeX, aby hlavičky kapitol
% sázel příčetněji a nevynechával nad nimi spoustu místa. Směle ignorujte.
\makeatletter
\def\@makechapterhead#1{
  {\parindent \z@ \raggedright \sffamily
   \Huge\bfseries \thechapter. #1
   \par\nobreak
   \vskip 20\p@
}}
\def\@makeschapterhead#1{
  {\parindent \z@ \raggedright \sffamily
   \Huge\bfseries #1
   \par\nobreak
   \vskip 20\p@
}}
\makeatother

% Trochu volnější nastavení dělení slov, než je default.
\lefthyphenmin=2
\righthyphenmin=2

% Zapne černé "slimáky" na koncích řádků, které přetekly, abychom si
% jich lépe všimli.
\overfullrule=1mm

%% Balíček hyperref, kterým jdou vyrábět klikací odkazy v PDF,
%% ale hlavně ho používáme k uložení metadat do PDF (včetně obsahu).
%% Většinu nastavítek přednastaví balíček pdfx.
\hypersetup{unicode}
\hypersetup{breaklinks=true}
\hypersetup{hidelinks}

%%% Prostředí pro sazbu kódu, případně vstupu/výstupu počítačových
%%% programů. (Vyžaduje balíček listings -- fancy verbatim.)
\lstnewenvironment{code}{\lstset{basicstyle=\small, frame=single}}{}



