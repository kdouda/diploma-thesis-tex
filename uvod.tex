\chapter*{Introduction}
\addcontentsline{toc}{chapter}{Introduction}

Traditionally, the study of animal movements and life cycles has been a domain of great uncertainty, owing to varying levels of difficulty in observing the full range of animal behaviour in the wild. This has been especially difficult for ornithologists studying migratory birds. With the advent of small electronic circuitry, global telecommunications and positioning systems, it has become easier to acquire the required primary data for study. With the increasing sophisticatedness of GPS loggers, energy efficient solar panels, batteries and faster communication networks, another common problem has arisen, especially for larger studies. Modern animal movement loggers can hundreds, if not thousands, of positions per day, multiple times. It has become increasingly difficult for field experts to make sense of their data manually, while comprehensive data analysis providing useful results is not a trivial task.

Speedy detection of animal life events, such as mortality or nesting, is crucial for actionable instructions for animal conservation fieldwork experts. For example, speedy detection of animal death is very important for helping to establish the cause of the mortality event, since specific causes will be more difficult to establish later. Other life events

The goal of this diploma thesis is to provide animal conservation experts with reliable methods of filtering unreliable data, simplifying or clustering too complex datasets for interpretation and finally, detecting important life events from these filtered and clustered data. The structure of this diploma thesis roughly follows the CRISP-DM methodology, and the resulting models will be released on GitHub and integrated into the Anitra platform for animal conservation experts.
