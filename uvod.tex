\chapter*{Introduction}
\addcontentsline{toc}{chapter}{Introduction}

Traditionally, the study of animal movements and life cycles has been a domain of great uncertainty, owing to varying levels of difficulty in observing the full range of animal behaviour in the wild. This has been especially difficult for ornithologists studying migratory birds. With the advent of small electronic circuitry, global telecommunications and positioning systems, it has become easier to acquire the required primary data for study using lightweight animal tracking devices, known as loggers or tags \cite{kays2015terrestrial}. These loggers collect various metrics, most importantly position at a certain time, from the animal carrying it.

During the infancy of the animal telemetry field, it was usually possible to analyse the received data manually due to its sparsity and precisely scheduled frequency, but required near-direct observation of the animal and had a great degree of inaccuracy \cite{burger2008perspectives}. With the increasing sophisticatedness of GPS loggers, energy efficient solar panels, batteries and faster communication networks \cite{kays2015terrestrial}, another common problem has arisen, especially for larger studies. Modern animal movement loggers can collect hundreds, if not thousands, of positions per day at very fine temporal scales \cite{gupte2022guide}, and send them to the end user multiple times per day, in any order. It has become increasingly difficult for field experts to make sense of their data manually, while comprehensive data analysis providing useful results is not a trivial task. Furthermore, with the increased data frequencies and volume, it has also been more difficult to detech and classify errorneous positions as such \cite{gupte2022guide}. 

Speedy detection of animal life events, such as mortality or nesting, is crucial for actionable instructions to animal conservation fieldwork experts. Speedy detection of the tracked animal's death is especially important for helping to establish the cause of the mortality event, since specific causes will be more difficult to establish with the passage of time.

The goal of this diploma thesis is to provide animal conservation experts, namely ornithologists, with reliable methods of filtering unreliable data, simplifying or clustering too complex datasets for interpretation, interpolating or subsetting data to satisfy model constraints, and finally, detecting important life events from these filtered and clustered data. The structure of this diploma thesis will follow the CRISP-DM methodology, and the resulting models will be released on GitHub and integrated into the Anitra platform for animal conservation experts. Users of this platform, notably members of the LIFE Eurokite project will be consulted for evaluation of these final models. Methods likely used for this project will include basic rule-based filtering, soft (fuzzy) clustering, hierarchical clustering, long-short term memory networks and possibly hidden Markov models. Various options will be evaluated based on test and train datasets and all final outputs will be evaluated by the previously mentioned field experts.
